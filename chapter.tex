%%%%%%%%%%%%%%%%%%%% author.tex %%%%%%%%%%%%%%%%%%%%%%%%%%%%%%%%%%%
%
% sample root file for your "contribution" to a contributed volume
%
% Use this file as a template for your own input.
%
%%%%%%%%%%%%%%%% Springer %%%%%%%%%%%%%%%%%%%%%%%%%%%%%%%%%%


% RECOMMENDED %%%%%%%%%%%%%%%%%%%%%%%%%%%%%%%%%%%%%%%%%%%%%%%%%%%
\documentclass[graybox]{svmult}

% choose options for [] as required from the list
% in the Reference Guide

\usepackage{mathptmx}       % selects Times Roman as basic font
\usepackage{helvet}         % selects Helvetica as sans-serif font
\usepackage{courier}        % selects Courier as typewriter font
\usepackage{type1cm}        % activate if the above 3 fonts are
                            % not available on your system
%
\usepackage{makeidx}         % allows index generation
\usepackage{graphicx}        % standard LaTeX graphics tool
                             % when including figure files
\usepackage{multicol}        % used for the two-column index
\usepackage[bottom]{footmisc}% places footnotes at page bottom

% see the list of further useful packages
% in the Reference Guide

\makeindex             % used for the subject index
                       % please use the style svind.ist with
                       % your makeindex program

%%%%%%%%%%%%%%%%%%%%%%%%%%%%%%%%%%%%%%%%%%%%%%%%%%%%%%%%%%%%%%%%%%%%%%%%%%%%%%%%%%%%%%%%%

\usepackage[utf8]{inputenc}
\usepackage[T1]{fontenc}




\usepackage{amsmath,amssymb}


%% Commands

\newcommand{\noun}[1]{\textsc{#1}}


%% Math

% Operators
\DeclareMathOperator{\Cov}{Cov}
\DeclareMathOperator{\Var}{Var}
\DeclareMathOperator{\Proba}{\mathbb{P}}

\newcommand{\Covb}[2]{\ensuremath{\Cov\!\left[#1,#2\right]}}
\newcommand{\Eb}[1]{\ensuremath{\E\!\left[#1\right]}}
\newcommand{\Pb}[1]{\ensuremath{\Proba\!\left[#1\right]}}
\newcommand{\Varb}[1]{\ensuremath{\Var\!\left[#1\right]}}
\newcommand{\rhob}[2]{\ensuremath{\rho\!\left[#1,#2\right]}}

% norm
\newcommand{\norm}[1]{\left\lVert #1 \right\rVert}



% argmin
\DeclareMathOperator*{\argmin}{\arg\!\min}



% renew graphics command for relative path providment only ?
%\renewcommand{\includegraphics[]{}}


\usepackage{url}

\def \draft {1}

\usepackage{xparse}
\usepackage{ifthen}
\DeclareDocumentCommand{\comment}{o m o o o o}
{\ifthenelse{\draft=1}{
  \IfValueT{#1}{
      \textcolor{red}{\textbf{C (#1) : }#2}
      \IfValueT{#3}{\textcolor{blue}{\textbf{A1 : }#3}}
      \IfValueT{#4}{\textcolor{green}{\textbf{A2 : }#4}}
      \IfValueT{#5}{\textcolor{red!50!blue}{\textbf{A3 : }#5}}
      \IfValueT{#6}{\textcolor{blue}{\textbf{A4 : }#6}}
    }
    \IfNoValueT{#1}{
      \textcolor{red}{\textbf{C : }#2}
      \IfValueT{#3}{\textcolor{blue}{\textbf{A1 : }#3}}
      \IfValueT{#4}{\textcolor{green}{\textbf{A2 : }#4}}
      \IfValueT{#5}{\textcolor{red!50!blue}{\textbf{A3 : }#5}}
      \IfValueT{#6}{\textcolor{blue}{\textbf{A4 : }#6}}
    }
 }{}
}


% geometry
\usepackage[margin=2cm]{geometry}

% layout : use fancyhdr package
%\usepackage{fancyhdr}
%\pagestyle{fancy}
%
%\makeatletter
%
%\renewcommand{\headrulewidth}{0.4pt}
%\renewcommand{\footrulewidth}{0.4pt}
%\fancyhead[RO,RE]{}
%\fancyhead[LO,LE]{Models for the co-evolution of cities and networks}
%\fancyfoot[RO,RE] {\thepage}
%\fancyfoot[LO,LE] {}
%\fancyfoot[CO,CE] {}
%
%\makeatother
%

%%%%%%%%%%%%%%%%%%%%%
%% Begin doc
%%%%%%%%%%%%%%%%%%%%%






\begin{document}

\title*{Perspectives on urban theories}
%\titlerunning{Relating complexities} 
\author{Denise Pumain, Juste Raimbault, Arnaud Banos}
% Use \authorrunning{Short Title} for an abbreviated version of
% your contribution title if the original one is too long
\institute{Juste Raimbault \at UPS CNRS 3611 ISC-PIF and CASA, UCL and UMR CNRS 8504 G{\'e}ographie-cit{\'e}s\\\medskip
Corresponding author: \texttt{juste.raimbault@polytechnique.edu}\\\medskip
Denise Pumain \at UMR CNRS 8504 G{\'e}ographie-cit{\'e}s\\\medskip
Arnaud Banos \at UMR CNRS IDEES
}

\maketitle


\abstract{Urban systems intrinsically need multiple complementary theories to be understood. This concluding chapter synthesizes the approaches developed in the book and aims at showing their complementarity and how new approaches making bridges can be developed. We first provide a broad overview of the theories and disciplines that the viewpoints developed in the book link to. We then extend this synthesis through citation network analysis, reconstructing a citation network from the references cited by authors, and identify disciplines from community detection within this network what allows to discuss the relative positioning of approaches. We finally discuss how modeling and simulation could be a systematic entry to the coupling of theories, and recall best practices in this particular context of building integrative complex simulation models of social systems.
\medskip\\
\textbf{Keywords : }\textit{Urban theories; Urban models; Simulation model coupling}
}



%%%%%%%%%%%%%%%%%%%%
\section{Introduction}





%%%%%%%%%%%%%%%%%%%%%%%%
\section{Complementary urban theories}

\textit{section par Denise : synthesis of the book, thematic link between theories used by the different chapters, positioning of disciplines and their relations (can insist on sociological aspects of science, as citation network analysis in the second section is a very reductionist view of the question)}



%%%%%%%%%%%%%%%%%%%%%%%%
\section{Citation network analysis}

We now turn to a quantitative analysis of the relative positioning of disciplines and approaches discussed above. We propose to use citation network analysis as a proxy to understand the structure of that scientific environment, what captures a single dimension of practices but contains relevant information on endogenous disciplinary structures. We use the method of \cite{} to construct a citation network at depth two, from the references cited by chapters of this book.

The bibliography of each chapter was manually indexed to ensure correct citing references retrieval during the data collection process. Furthermore, for performance purposes, but also to ensure a focus of the network content on urban issues, references clearly out of the scope and which would yield a significant part of the final network totally unrelated to urban theories (the paper on morphogenesis \cite{} is a typical example, being anectdotically cited by papers relating to urban issues, but also massively cited by several branches of biology). The corpus used is available as raw data, along with the code and results, on the open git repository of the project at \url{}.





%%%%%%%%%%%%%%%%%%%%%%%%
\section{Modeling and simulation}

\textit{section par Arnaud : version anglaise des 9 commandements de Banos pour la modelisation et simulation en SHS}

\comment[JR]{suggestions :
\begin{itemize}
	\item specificité des commandements dans le cas de la geographie ?
	\item proposition à amender / discuter selon cotre positionnement : le couplage des modeles (de simulation) et donc la construction de modeles intégrés est une façon robuste de coupler les théories - d'où l'intérêt des commandements dans le contexte de ponts entre les differentes theories urbaines ? (l'intérêt du couplage en lui meme est déjà présent ici)
\end{itemize}
}





%%%%%%%%%%%%%%%%%%%%
\section{Discussion}






%%%%%%%%%%%%%%%%%%%%
\section*{Conclusion}
%%%%%%%%%%%%%%%%%%%%






%%%%%%%%%%%%%%%%%%%%
%% Biblio
%%%%%%%%%%%%%%%%%%%%

%\footnotesize

%\begin{multicols}{2}
\bibliographystyle{apalike}
\bibliography{biblio}
%\end{multicols}


%\begin{thebibliography}{}
%\end{thebibliography}





\end{document}
